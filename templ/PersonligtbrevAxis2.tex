\documentclass[]{article}
\usepackage{lmodern}
\usepackage{amssymb,amsmath}
\usepackage{ifxetex,ifluatex}
\usepackage{fixltx2e} % provides \textsubscript
\ifnum 0\ifxetex 1\fi\ifluatex 1\fi=0 % if pdftex
  \usepackage[T1]{fontenc}
  \usepackage[utf8]{inputenc}
\else % if luatex or xelatex
  \ifxetex
    \usepackage{mathspec}
  \else
    \usepackage{fontspec}
  \fi
  \defaultfontfeatures{Ligatures=TeX,Scale=MatchLowercase}
\fi
% use upquote if available, for straight quotes in verbatim environments
\IfFileExists{upquote.sty}{\usepackage{upquote}}{}
% use microtype if available
\IfFileExists{microtype.sty}{%
\usepackage{microtype}
\UseMicrotypeSet[protrusion]{basicmath} % disable protrusion for tt fonts
}{}
\usepackage[margin=1in]{geometry}
\usepackage{hyperref}
\hypersetup{unicode=true,
            pdftitle={Cover letter Axis},
            pdfborder={0 0 0},
            breaklinks=true}
\urlstyle{same}  % don't use monospace font for urls
\usepackage{graphicx,grffile}
\makeatletter
\def\maxwidth{\ifdim\Gin@nat@width>\linewidth\linewidth\else\Gin@nat@width\fi}
\def\maxheight{\ifdim\Gin@nat@height>\textheight\textheight\else\Gin@nat@height\fi}
\makeatother
% Scale images if necessary, so that they will not overflow the page
% margins by default, and it is still possible to overwrite the defaults
% using explicit options in \includegraphics[width, height, ...]{}
\setkeys{Gin}{width=\maxwidth,height=\maxheight,keepaspectratio}
\IfFileExists{parskip.sty}{%
\usepackage{parskip}
}{% else
\setlength{\parindent}{0pt}
\setlength{\parskip}{6pt plus 2pt minus 1pt}
}
\setlength{\emergencystretch}{3em}  % prevent overfull lines
\providecommand{\tightlist}{%
  \setlength{\itemsep}{0pt}\setlength{\parskip}{0pt}}
\setcounter{secnumdepth}{0}
% Redefines (sub)paragraphs to behave more like sections
\ifx\paragraph\undefined\else
\let\oldparagraph\paragraph
\renewcommand{\paragraph}[1]{\oldparagraph{#1}\mbox{}}
\fi
\ifx\subparagraph\undefined\else
\let\oldsubparagraph\subparagraph
\renewcommand{\subparagraph}[1]{\oldsubparagraph{#1}\mbox{}}
\fi

%%% Use protect on footnotes to avoid problems with footnotes in titles
\let\rmarkdownfootnote\footnote%
\def\footnote{\protect\rmarkdownfootnote}

%%% Change title format to be more compact
\usepackage{titling}

% Create subtitle command for use in maketitle
\newcommand{\subtitle}[1]{
  \posttitle{
    \begin{center}\large#1\end{center}
    }
}

\setlength{\droptitle}{-2em}
  \title{Cover letter Axis}
  \pretitle{\vspace{\droptitle}\centering\huge}
  \posttitle{\par}
  \author{}
  \preauthor{}\postauthor{}
  \date{}
  \predate{}\postdate{}


\begin{document}
\maketitle

\begin{center}\rule{0.77\paperwidth}{\linethickness}\end{center}
%{0.5\linewidth}{\linethickness}\end{center}

\textbf{Hampus Hjelm Andersson}\\
Gärdesgata 13, 33533, Gnosjö\\
\href{https://github.com/HHA123/Job/}{Github Account}
\textbar{}\textbar{}
\href{mailto:hampusha@hotmail.com}{\nolinkurl{hampusha@hotmail.com}}
\textbar{}\textbar{} (+46) 0708282487

June 10, 2018\\
Reference number: 2274

\begin{center}\rule{0.77\paperwidth}{\linethickness}\end{center}

\hypertarget{section}{%
\section{\texorpdfstring{\\
}{ }}\label{section}}

Dear Hiring Manager

I am writing to you to express my Great interest for the position of
``Image technology engineer on Product Platforms''.

My brother Alexander Hjelm Bankers works on your company and recommends
it highly as a very good place to work.

\begin{itemize}
\item
  I have a Masters in Mathematics from Linköpings University and the
  greatest asset I can bring to your company from my studies in
  mathematics would to be able to understand the meaning of mathematical
  formula and to generalize it to new problems, this would probably be
  very useful to you in creating new products.
\item
  I also have a great interests in (convolution) neural networks and
  it's application to image processing, you can see a small exampel of
  an implementation I have done in digit recognition in Tensorflow
  \href{https://github.com/HHA123/Job/blob/master/TensorflowNNdigit.ipynb}{here}.
\item
  Further more I have some experience with
  \href{https://github.com/HHA123/Job/blob/master/RepRes2.html}{data
  processing} and
  \href{https://github.com/HHA123/Job/blob/master/Statinf2.Rmd}{statistics}
  in R
\item
  My experience in meditation can help me better to coup with stress and
  to let go of old ways of thinking which I think can be very helpful
  for having an experimental way of thinking which could help you to
  develop new functionalities.
\item
  I am very eager to learn more about C , convulotional neural networks
  and other imageprocessing algorithms.
\end{itemize}

I would greatly appreciate the opportunity to meet with you to discuss
what I have to bring to the position at Axis\\
for instances, I can already think of some suggestions for
functionalities to your surveillance cameras\\

\begin{itemize}
\tightlist
\item
  a functionality to choose so that some people are ``allowable'' and
  others not.
\item
  a functionality so that the user can choose to use an algorithm in a
  limited field of visible space, for instances if you have a backyard
  towards a street you might want to register if someone goes into the
  backyard but exclude the area of the street.
\item
  if the cameras are connected to the internet one could have servers
  that the cameras can connect to with gpu:s that could make more
  advanced calculations then would be possible to do on a single camera
\item
  making a functionality so that the user can choose what objects to
  detect: e.g.~say if someone have a graffiti problem, then the user
  should be able to add algorithms to detect spray cans\ldots{}
\item
  making sure that algorithms are user friendly
\end{itemize}

Thank you for your time and consideration

Sincerely,

Hampus Hjelm Andersson


\end{document}
