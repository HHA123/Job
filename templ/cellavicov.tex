\documentclass[]{article}
\usepackage{lmodern}
\usepackage{amssymb,amsmath}
\usepackage{graphicx}
\usepackage{float}
\usepackage{ifxetex,ifluatex}
\usepackage{fixltx2e} % provides \textsubscript
\ifnum 0\ifxetex 1\fi\ifluatex 1\fi=0 % if pdftex
  \usepackage[T1]{fontenc}
  \usepackage[utf8]{inputenc}
\else % if luatex or xelatex
  \ifxetex
    \usepackage{mathspec}
  \else
    \usepackage{fontspec}
  \fi
  \defaultfontfeatures{Ligatures=TeX,Scale=MatchLowercase}
\fi
% use upquote if available, for straight quotes in verbatim environments
\IfFileExists{upquote.sty}{\usepackage{upquote}}{}
% use microtype if available
\IfFileExists{microtype.sty}{%
\usepackage{microtype}
\UseMicrotypeSet[protrusion]{basicmath} % disable protrusion for tt fonts
}{}
\usepackage[margin=1in]{geometry}
\usepackage{hyperref}
\hypersetup{unicode=true,
            pdftitle={Cover letter CellaVision},
            pdfborder={0 0 0},
            breaklinks=true}
\urlstyle{same}  % don't use monospace font for urls
\usepackage{graphicx,grffile}
\makeatletter
\def\maxwidth{\ifdim\Gin@nat@width>\linewidth\linewidth\else\Gin@nat@width\fi}
\def\maxheight{\ifdim\Gin@nat@height>\textheight\textheight\else\Gin@nat@height\fi}
\makeatother
% Scale images if necessary, so that they will not overflow the page
% margins by default, and it is still possible to overwrite the defaults
% using explicit options in \includegraphics[width, height, ...]{}
\setkeys{Gin}{width=\maxwidth,height=\maxheight,keepaspectratio}
\IfFileExists{parskip.sty}{%
\usepackage{parskip}
}{% else
\setlength{\parindent}{0pt}
\setlength{\parskip}{6pt plus 2pt minus 1pt}
}
\setlength{\emergencystretch}{3em}  % prevent overfull lines
\providecommand{\tightlist}{%
  \setlength{\itemsep}{0pt}\setlength{\parskip}{0pt}}
\setcounter{secnumdepth}{0}
% Redefines (sub)paragraphs to behave more like sections
\ifx\paragraph\undefined\else
\let\oldparagraph\paragraph
\renewcommand{\paragraph}[1]{\oldparagraph{#1}\mbox{}}
\fi
\ifx\subparagraph\undefined\else
\let\oldsubparagraph\subparagraph
\renewcommand{\subparagraph}[1]{\oldsubparagraph{#1}\mbox{}}
\fi

%%% Use protect on footnotes to avoid problems with footnotes in titles
\let\rmarkdownfootnote\footnote%
\def\footnote{\protect\rmarkdownfootnote}

%%% Change title format to be more compact
\usepackage{titling}

% Create subtitle command for use in maketitle
\newcommand{\subtitle}[1]{
  \posttitle{
    \begin{center}\large#1\end{center}
    }
}

\setlength{\droptitle}{-2em}
  \title{Cover letter\vspace{1cm}} %CellaVision}
%\begin{figure}[H]
%\centering
%\end{figure}

  \pretitle{\scalebox{0.9}[0.8]{\includegraphics{cellavi.png}}
\vspace{\droptitle}\centering\huge}
  \posttitle{\par}
  \author{}
  \preauthor{}\postauthor{}
  \date{}
  \predate{}\postdate{}


\begin{document}
%\maketitle



\maketitle

\begin{center}\rule{0.99\linewidth}{\linethickness}\end{center}
%0.77
\textbf{Hampus Hjelm Andersson}\\
Gärdesgata 13, 33533, Gnosjö\\
\href{https://github.com/HHA123/Job/}{\textbf{Github Account:github.com/HHA123/Job}
}
\textbar{}\textbar{}
\href{mailto:hampusha@hotmail.com}{\nolinkurl{hampusha@hotmail.com}}
\textbar{}\textbar{} (+46) 0708282487 \\
February 6, 2019
\vspace{1mm}
\begin{center}\rule{0.99\linewidth}{\linethickness}\end{center}
%0.77
\hypertarget{section}{%
\section{\texorpdfstring{\\
}{ }}\label{section}}

Dear Hiring Manager

I am writing to you to express my Great interest for the position of
"Junior Software engineer".

I saw your recruitment add on your website and found it very interesting.
I particularly like that you use image analysis and neural networks
 to create products that  help people.
 I have my self been training neural networks and convolutional neural networks for
 the last 6 months and would like to continue with
 this in a professional setting.
The things I might be able to contribute to your company would be.




%My brother Alexander Hjelm Bankers works on your company and recommends
%it highly as a very good place to work.

\begin{itemize}
\item
  I have a Masters in Mathematics from Linköpings University, the
  experience that I have from these studies in
  mathematics help me to be able to understand the meaning of mathematical
  formula and to generalize it to new problems.
%, this would probably be  very useful to you in creating new products.
\item
  I also have a great interests in (convolution) neural networks and
  it's application to image processing, you can see a small example of
  an implementation I have done in digit recognition in Tensorflow
  \href{https://github.com/HHA123/Job/blob/master/TensorflowNNdigit.ipynb}{\textbf{here}}.

\item
  To illustrate some of what I've learnt in the training of
 convolutional neural networks, I tried solving the histopathologic-cancer-detection challenge
 at kaggle.com, where one is asked to predict whether there is cancer
 in images of lymph node sections. To solve this problem with over 220 thousand
pictures I had to use some common data techniques such as, data generators,
 splitting data in training, validation and test set, data augmentation,
transfer learning, fixing initial layers after some time of training,
 test time augmentation was also used on the test set.
 When submitted to Kaggle this yielded a result of 97.07\% only 0.86 percentile
 from the highest result.
You can see the jupyter notebook \href{https://github.com/HHA123/Job/blob/master/histo-pat-cancer.ipynb}{\textbf{here}}

  

\item
 Some of my favorite courses during University were Matrix analysis, Linear algebra
honors course and linear algebra which
I think can help to make it easier to understand what
 different layers in the convolutional does. And I also believe it could help to develop
new models.

 


\item
  Further more I have some experience with
 % \href{https://github.com/HHA123/Job/blob/master/RepRes2.html}
\href{https://github.com/HHA123/Job/blob/master/RepRes2.html}
{\textbf{data processing}} and
  \href{https://github.com/HHA123/Job/blob/master/Statinf2.Rmd}{\textbf{statistics}}
  in R
%\item
 % My experience in meditation can help me better to coup with stress and
 % to let go of old ways of thinking which I think can be very helpful
 % for having an experimental way of thinking which could help you to
 % develop new functionalities.


\item
  The things I primary would like to work with professionally is data processing
, training and test of different kinds of neural networks and visualization of
the result for domain experts, i.e class activation maps or other
 simlilar techniques.



%\item
%  I am very eager to learn more about C , convulotional neural networks
%  and other imageprocessing algorithms.
\end{itemize}

I would greatly appreciate the opportunity to meet with you to discuss
what I have to bring to the position at CellaVision\\
%for instances, I can already think of some suggestions for
%functionalities to your surveillance cameras\\

%\begin{itemize}
%\tightlist
%\item
%  a functionality to choose so that some people are ``allowable'' and
%  others not.
%\item
%  a functionality so that the user can choose to use an algorithm in a
%  limited field of visible space, for instances if you have a backyard
%  towards a street you might want to register if someone goes into the
%  backyard but exclude the area of the street.
%\item
%  if the cameras are connected to the internet one could have servers
%  that the cameras can connect to with gpu:s that could make more
%  advanced calculations then would be possible to do on a single camera
%\item
%  making a functionality so that the user can choose what objects to
%  detect: e.g.~say if someone have a graffiti problem, then the user
%  should be able to add algorithms to detect spray cans\ldots{}
%\item
%  making sure that algorithms are user friendly
%\end{itemize}

Thank you for your time and consideration

Sincerely,

Hampus Hjelm Andersson


\end{document}
